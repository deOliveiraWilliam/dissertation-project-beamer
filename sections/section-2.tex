\section{Definição do Problema}

\subsection{Movimentos de Massa}
\begin{frame}{Movimentos de Massa}
    Tipos de movimentos:
    \vspace{.25cm}
    \begin{itemize}
        \item Desmoronamentos;
        \item Tombamentos;
        \item Deslizamento;
        \item Corridas detríticas;
        \item Corridas de lama.
    \end{itemize}
\end{frame}

\begin{frame}{Movimentos de Massa}
    \begin{table}[ht]
        \centering
        \captionof{table}{Magnitude e mobilidade dos movimentos de massa\footnote{Fonte: \cite{takahashi_debris_2014}}.}
        \small
        \begin{tabular}{lccc}
                \hline
                Fenômeno                & Volume deslocado                                             & Velocidade                             & Distância máxima \\
                                        & (\SI{}{\cubic\meter})                                & (\unit[per-mode = symbol]{\metre\per\second})                                 & (\SI{}{\kilo \meter})             \\ \hline
                Deslizamento de terra   & 10 $\sim$ 10\textsuperscript{6}                       & 10\textsuperscript{-6} $\sim$ 10       & $<$ 0,3          \\
                Desmoronamento          & 2 $\times$ 10\textsuperscript{5}                      & \textemdash                                     & 0,7              \\
                Corridas Detríticas     & 10\textsuperscript{3} $\sim$ 10\textsuperscript{6}    & 0,5 $\sim$ 20                          & 0,2 $\sim$ 10    \\
                Avalanches Detríticas   & 10\textsuperscript{7} $\sim$ 10\textsuperscript{10}   & 10 $\sim$ 10\textsuperscript{2}        & < 30             \\
                    \hline
            \end{tabular}
            \label{tab:mov_massa_magnitude}
        \end{table}
\end{frame}

\subsection{Classificação dos Movimentos}

\begin{frame}{Movimentos de Massa}
    \begin{minipage}[c]{0.70\textwidth}
        \centering
        \includegraphics[width=\textwidth]{../dissertation-project/fig/diagrams/rheological_classification_english.pdf}
    \end{minipage}
    \hfill    
    \begin{minipage}[c]{0.28\textwidth}
        \captionof{figure}{Classificação dos movimentos de massa em encostas íngremes, como função
        da porção sólida e do tipo de material. Adaptado de \cite{coussot_recognition_1996}.}
    \end{minipage}
\end{frame}


\subsection{Escoamento em Canais}

\begin{frame}{Escoamento em canais}
    \begin{itemize}
        \item Canal aberto (superfície livre)
    \end{itemize}    
    \begin{minipage}[c]{0.50\textwidth}
        \centering
        \includegraphics[width=\textwidth]{../dissertation-project/fig/diagrams/channel_section.pdf}
        \captionof{figure}{Diagrama com seção transversal de um canal genérico.}
    \end{minipage}
    \hfill
    \pause
    \begin{minipage}[c]{0.36\textwidth}
        \begin{itemize}
            \item Seção Retangular
        \end{itemize}
            \vspace{0.25cm}
            \centering
            \includegraphics[width=0.5\textwidth]{../dissertation-project/fig/diagrams/cross_section_rectangular.pdf}
            \captionof{figure}{Canal retangular.}
        \vspace{0.50cm}
        
        \begin{itemize}
            \item Raio hidráulico
        \end{itemize}
        \vspace{0.25cm}
        \begin{equation}
            R_h = \frac{Bh}{B+2h}
        \end{equation}
    \end{minipage}
\end{frame}

\subsection{Hipóteses de Simplificação}

\begin{frame}{Hipóteses de Simplificação}
    \begin{itemize}
        \item Escoamento uniforme;
        \item Escoamento laminar;
        \item Canal aberto;
        \item Teoria de águas rasas;
        \item Canal infinitamente largo.
    \end{itemize}
\end{frame}
