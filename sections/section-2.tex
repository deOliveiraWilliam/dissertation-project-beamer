\section{Fundamentação}

\subsection{Movimentos de Massa}
% \begin{frame}{Movimentos de Massa}
%     \begin{exampleblock}{\Large Tipos de movimentos}
%         \begin{itemize}
%             \Large
%             \item Desmoronamentos;
%             \item Tombamentos;
%             \item Deslizamento de terra;
%             \item Corridas detríticas;
%             \item Corridas de lama.
%         \end{itemize}
%     \end{exampleblock}
% \end{frame}

\begin{frame}{Movimentos de Massa}
    \begin{minipage}[c]{0.70\textwidth}
        \centering
        \includegraphics[width=\textwidth]{../dissertation-project/fig/diagrams/rheological_classification_english.pdf}
    \end{minipage}
    \hfill    
    \begin{minipage}[c]{0.28\textwidth}
        \captionof{figure}{Classificação dos movimentos de massa em encostas íngremes, como função
        da porção sólida e do tipo de material. Adaptado de \cite{coussot_recognition_1996}.}
    \end{minipage}
\end{frame}

\begin{frame}{Movimentos de Massa}
    \begin{table}[ht]
        \centering
        \captionof{table}{Magnitude e mobilidade dos movimentos de massa\footnote{Fonte: \cite{takahashi_debris_2014}}.}
        \small
        \begin{tabular}{lccc}
                \hline
                Fenômeno                & Volume deslocado                                             & Velocidade                             & Distância máxima \\
                                        & (\SI{}{\cubic\meter})                                & (\unit[per-mode = symbol]{\metre\per\second})                                 & (\SI{}{\kilo \meter})             \\ \hline
                Deslizamento de terra   & 10 $\sim$ 10\textsuperscript{6}                       & 10\textsuperscript{-6} $\sim$ 10       & $<$ 0,3          \\
                Desmoronamento          & 2 $\times$ 10\textsuperscript{5}                      & \textemdash                                     & 0,7              \\
                Corridas Detríticas     & 10\textsuperscript{3} $\sim$ 10\textsuperscript{6}    & 0,5 $\sim$ 20                          & 0,2 $\sim$ 10    \\
                Avalanches Detríticas   & 10\textsuperscript{7} $\sim$ 10\textsuperscript{10}   & 10 $\sim$ 10\textsuperscript{2}        & < 30             \\
                    \hline
            \end{tabular}
            \label{tab:mov_massa_magnitude}
        \end{table}
\end{frame}



\subsection{Escoamento em Canais}

\begin{frame}{Escoamento em canais}
    \begin{minipage}[c]{0.56\textwidth}
        \begin{exampleblock}{Canal aberto (superfície livre)}
            \centering
            \includegraphics[width=\textwidth]{../dissertation-project/fig/diagrams/channel_section.pdf}
            \captionof{figure}{Diagrama com seção transversal de um canal genérico.}           
        \end{exampleblock}
    \end{minipage}
    \hfill
    \pause
    \begin{minipage}[c]{0.36\textwidth}
        \begin{exampleblock}{Seção Retangular}
            \centering
            \includegraphics[width=0.5\textwidth]{../dissertation-project/fig/diagrams/cross_section_rectangular.pdf}
            \captionof{figure}{Seção canal retangular.}    
        \end{exampleblock}
        \begin{exampleblock}{Raio hidráulico}
            \begin{equation}
                R_h = \frac{Bh}{B+2h}
            \end{equation}
        \end{exampleblock}
    \end{minipage}
\end{frame}

\begin{frame}{Adimensionais}
    \begin{exampleblock}{Número de Froude}
        \begin{equation}
            Fr = \frac{U}{\sqrt{gh}} 
            = \frac{\text{Forças Inerciais}}{\text{Forças Gravitacionais}}
        \end{equation}        
    \end{exampleblock}

    \begin{exampleblock}{Número de Reynolds}
        \begin{equation}
            Re = \frac{\rho U L}{\mu} = \frac{4UR_h}{\nu} 
            = \frac{\text{Forças Inerciais}}{\text{Forças Viscosas}}
        \end{equation}        
    \end{exampleblock}
\end{frame}



\subsection{Reologia}

\begin{frame}{Reologia}
    \includegraphics[width=\textwidth]{../dissertation-project/fig/diagrams/fluid_classification.pdf}
    \captionof{figure}{Fluidos não newtonianos. (a) Curvas se fluidos
    viscoplásticos. (b) Fluidos Puramente Viscosos. (c) Teste de taxa de deformação
    constante.}
\end{frame}

\begin{frame}{Reologia}
    \centering
    \includegraphics[width=0.65\textwidth]{../dissertation-project/fig/diagrams/water-debris_classification.pdf}
    \captionof{figure}{Classificação reológica de misturas de água e detritos. Adaptado de \cite{coussot_1997_mudflow}.}
\end{frame}

\subsection{Equacionamento}
\begin{frame}
    \begin{exampleblock}{Conservação da Massa}
        \begin{equation}
            \partialderiv{\rho}{t} + \nabla \cdot \rho \bm{u} = 0
        \end{equation}
    \end{exampleblock}

    \begin{exampleblock}{Conservação de \textit{Momentum}}
        \begin{equation}
            \rho \left( \partialderiv{\bm{u}}{t}  + \bm{u} \cdot \nabla \bm{u} \right) =
            \bm{b} + \nabla \cdot \bm{T}
        \end{equation}
    \end{exampleblock}

    \hspace{0.5cm} Definindo o tensor de tensão como $\bm{T} = -p\bm{I} + \bm{\tau}$ e a 
    aceleração gravitacional como a única força externa aplicada ao fluido:
        \vspace{0.25cm}
        \begin{equation}
            \rho \left( \partialderiv{\bm{u}}{t}  + \bm{u} \cdot \nabla \bm{u} \right) =
            -\nabla p + \nabla \cdot \bm{\tau} + \rho \bm{g}
        \end{equation}
\end{frame}

\begin{frame}
    \hspace{0.5cm} Em regime permanente, a equação de \textit{momentum} fica:
    \begin{equation}
        \begin{split}
            \partialderiv{\tau}{y} + \rho g \sin{\theta} &= 0 \\
            -\partialderiv{p}{y} - \rho g \cos{\theta} &= 0
        \end{split}
        \label{eq:motion_steady_state_partial}
    \end{equation}

    \hspace{0.5cm} Integrando e usando as condições de contorno da superfície livre e não deslizamento na parede:
    \begin{equation}
        \begin{split}
            u(y) &= \frac{\alpha}{m+1} \left[y_0^{(m+1)} - {(y_0-y)}^{(m+1)}\right]
            \hspace{0.5cm} \text{quando} \hspace{0.5cm} y \leq y_0
            \\
            u(y) &= u(y_0) = \frac{\alpha}{m+1} \left[ y_0^{(m+1)} \right]
            \hspace{1.9cm} \text{quando} \hspace{0.5cm} h \geq y \geq y_0
        \end{split}
        \label{eq:velocity_distribution}
    \end{equation}
    
    com $m=\frac{1}{n}$, $y_0 = h - \frac{\yieldstress}{\rho g \sin{\theta}}$ e $\alpha = \left({\frac{\rho g \sin{\theta}}{K}}\right)^m$.
\end{frame}

\begin{frame}
    \begin{minipage}[c]{0.65\textwidth}
        \begin{exampleblock}{Perfil de velocidade de um fluido tipo HB}
            \centering
            \includegraphics[width=0.85\textwidth]{../dissertation-project/fig/diagrams/HB_velocity-distribution_inclined-plane.pdf}
            \captionof{figure}{Distribuição de velocidade ao longo da seção transversal, para diferentes valores de $n$. Adaptado de \cite{coussot_1997_mudflow}.}
        \end{exampleblock}
    \end{minipage}
    \hfill
    \begin{minipage}[t]{0.32\textwidth}
        \small
        \begin{equation}
            \begin{split}
                U &= \frac{u(y)}{u(y_0)} \\
                Y &= \frac{y}{y_0} \\
                G &= \frac{\rho g \sin{\theta}}{\yieldstress}
            \end{split}
            \label{eq:dimensionaless_velocity_distribution}
        \end{equation}        
    \end{minipage}
\end{frame}

\subsection{Adimensionais}

\begin{frame}

    \begin{exampleblock}{Modelo de fluido tipo HB}
        \begin{equation}
            \label{eq:HB_shear_stress}
            \begin{split}
                \tau &= \yieldstress + K \shearrate^n
                \text{ ,\hspace{0.5 cm} quando \hspace{0.2cm}} \shearrate = \partialderiv{u}{y} \ne 0
                \\
                |\tau| &\leq \yieldstress
                \text{ ,\hspace{1.65cm} quando \hspace{0.2cm}} \shearrate = 0
            \end{split}
        \end{equation}
    \end{exampleblock}

    \begin{exampleblock}{Reynolds canal aberto, fluido tipo HB}
        \begin{equation}
            Re_{HB} = \frac{8 \rho U^2}{\tau_c + K \left( \frac{2U}{R_h} \right)^n}
        \end{equation}
    \end{exampleblock}

\end{frame}