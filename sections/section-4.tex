\section{Escoamento Sobre Estrutura}

% \begin{frame}{Características}
%     \Large
%     \begin{itemize}
%         \item Escoamento uniforme;
%         \item Escoamento laminar;
%         \item Superfície livre;
%         \item Canal infinitamente largo.
%     \end{itemize}
% \end{frame}

\subsection{Escoamento Sobre Estrutura}

\begin{frame}{Definição do Problema}
    \begin{minipage}[c]{0.49\textwidth}
        \begin{exampleblock}{Plano inclinado, bidimensional}
            \includegraphics[width=\textwidth]{../dissertation-project/fig/diagrams/free-surface_inclined-plane.pdf}
            \captionof{figure}{Escoamento de superfície livre de altura $h$ em um plano infinitamente largo, com
            velocidade uniforme.}
        \end{exampleblock}
    \end{minipage}
    \hfill
    \begin{minipage}[c]{0.49\textwidth}
        \begin{exampleblock}{Plano inclinado, "tridimensional"}
            \includegraphics[width=\textwidth]{../dissertation-project/fig/diagrams/free-surface-obstacle.pdf}
            \captionof{figure}{Escoamento incidindo sobre um corpo submerso retangular ao fundo do canal, com velocidade $u(y)$.}
        \end{exampleblock}
    \end{minipage}
\end{frame}

\begin{frame}{Literatura}
    
    \begin{exampleblock}{\Citeauthor{Forbes1982} (1982)}
        \begin{itemize}
            \item Modelo numérico, bidimensional;
            \item Fluido invíscido e incompressível;
            \item Mecanismo de geração de ondas.
        \end{itemize}
    \end{exampleblock}

    \begin{exampleblock}{\Citeauthor{Lawrence1987} (1987)}
        \begin{itemize}
            \item Aparato experimental, obstáculo fixo e submerso;
            \item Regimes hidráulicos dependentes de perturbação/formação escoamento.
        \end{itemize}
    \end{exampleblock}

\end{frame}

\begin{frame}{Literatura}
    \begin{exampleblock}{\Citeauthor{Vanden-Broeck1987} (1987); 
        \Citeauthor{King1987} (1987); \Citeauthor{Forbes1988} (1988)}
        \begin{itemize}
            \item Fluidos ideais;
            \item Obstáculos de diferentes formatos;
            \item Variação do número de Froude caracterizando o regime hidráulico.
        \end{itemize}
    \end{exampleblock}

    \begin{exampleblock}{\Citeauthor{Fadda1997} (1997)}
        \begin{itemize}
            \item Numérico e experimental;
            \item Dimensão característica do obstáculo;
            \item Variações de vazão, influência do número de Reynolds.
        \end{itemize}
    \end{exampleblock}
\end{frame}

\begin{frame}{Literatura}
    \begin{exampleblock}{\Citeauthor{Bush1994} (1994); \Citeauthor{Singh2000} (2000)}
        \begin{itemize}
            \item Fluidos viscoelásticos;
            \item Sedimentação em diferentes fluidos.
        \end{itemize}
    \end{exampleblock}

    \begin{exampleblock}{\Citeauthor{Bernabeu2018} (2018); \Citeauthor{Hinton2023} (2023)}
        \begin{itemize}
            \item Fluido viscoplástico;
            \item Reologias mais complexas;
            \item Parâmetros do fluido influenciando a dinâmica do escoamento.
        \end{itemize}
    \end{exampleblock}
\end{frame}


% \begin{frame}{Literatura}
%     \begin{minipage}[c]{0.49\textwidth}
%         \begin{exampleblock}{Primeiros estudos}
%             \begin{itemize}
%                 \item Validação de modelos lineares;
%                 \item Fluidos invíscidos, escoamento potencial;
%                 \item Domínio bidimensional;
%                 % \item Estudos numérico/experimental;
%                 \item Forma e disposição de obstáculos.
%             \end{itemize}
%         \end{exampleblock}
%     \end{minipage}
%     \hfill
%     \begin{minipage}[c]{0.49\textwidth}
%         \begin{exampleblock}{Trabalhos mais recentes}
%             \begin{itemize}
%                 \item Regimes hidráulicos subcrítico, crítico e supercrítico;
%                 % \item Avaliações numéricas/experimental;
%                 \item Aplicação de fluidos não newtonianos;
%                 \item Zonas de estagnação da velocidade;
%                 \item Efeitos dinâmicos do escoamento.
%             \end{itemize}
%         \end{exampleblock}
%     \end{minipage}
        
% \end{frame}

% \begin{frame}{Tópicos de Análise}
%     \begin{itemize}
%         \large
%         \item Comportamento da superfície livre;
%         \item Variação da vazão e seus efeitos na dinâmica do escoamento;
%         \item Coeficientes de arrasto e sustentação sob diferentes cenários;
%         \item Avaliação dos esforços induzidos no obstáculo submerso;
%         \item Influência da região de estagnação do escoamento;
%         \item Números de Reynolds e de Froude para caracterizar o escoamento.
%     \end{itemize}
% \end{frame}