\section{Escoamento Sobre Estrutura}

% \begin{frame}{Características}
%     \Large
%     \begin{itemize}
%         \item Escoamento uniforme;
%         \item Escoamento laminar;
%         \item Superfície livre;
%         \item Canal infinitamente largo.
%     \end{itemize}
% \end{frame}

\subsection{Escoamento Sobre Estrutura}

\begin{frame}{Definição do Problema}
    \begin{minipage}[c]{0.49\textwidth}
        \begin{exampleblock}{Plano inclinado, bidimensional}
            \includegraphics[width=\textwidth]{../dissertation-project/fig/diagrams/free-surface_inclined-plane.pdf}
            \captionof{figure}{Escoamento de superfície livre de altura $h$ em um plano infinitamente largo, com
            velocidade uniforme.}
        \end{exampleblock}
    \end{minipage}
    \hfill
    \begin{minipage}[c]{0.49\textwidth}
        \begin{exampleblock}{Plano inclinado, "tridimensional"}
            \includegraphics[width=\textwidth]{../dissertation-project/fig/diagrams/free-surface-obstacle.pdf}
            \captionof{figure}{Escoamento incidindo sobre um corpo submerso retangular ao fundo do canal, com velocidade $u(y)$.}
        \end{exampleblock}
    \end{minipage}
\end{frame}

% \begin{frame}
%     \begin{exampleblock}{Dinâmica do Escoamento}
%         \begin{itemize}
%             \item Comportamento da superfície livre;
%             \item Coeficientes de arrasto e sustentação;
%             \item Esforços sobre o obstáculo;
%             \item Vazões na entrada do domínio;
%             \item Forças em função da viscosidade do fluido.
%         \end{itemize}
%     \end{exampleblock}
% \end{frame}

% \begin{frame}
%     \begin{exampleblock}{Definições Gerais}
%         \begin{itemize}
%             \item Escoamento laminar;
%             \item Escoamento de superfície livre;
%             \item Canal de fundo fixo;
%             \item Diferentes tipos de obstáculos;
%             \item Análise numérica.
%         \end{itemize}    
%     \end{exampleblock}
% \end{frame}

% \subsection{Hipóteses de Simplificação}

\begin{frame}{Literatura}
    \begin{minipage}[c]{0.49\textwidth}
        \begin{exampleblock}{Primeiros estudos}
            \begin{itemize}
                \item Validação de modelos lineares;
                \item Fluidos invíscidos, escoamento potencial;
                \item Domínio bidimensional;
                % \item Estudos numérico/experimental;
                \item Forma e disposição de obstáculos.
            \end{itemize}
        \end{exampleblock}
    \end{minipage}
    \hfill
    \begin{minipage}[c]{0.49\textwidth}
        \begin{exampleblock}{Trabalhos mais recentes}
            \begin{itemize}
                \item Regimes hidráulicos subcrítico, crítico e supercrítico;
                % \item Avaliações numéricas/experimental;
                \item Aplicação de fluidos não newtonianos;
                \item Zonas de estagnação da velocidade;
                \item Efeitos dinâmicos do escoamento.
            \end{itemize}
        \end{exampleblock}
    \end{minipage}
        
\end{frame}

\begin{frame}{Objetivos específicos}
    \begin{itemize}
        \large
        \item Análise do comportamento da superfície livre;
        \item Determinação dos coeficientes de arrasto e sustentação em diferentes cenários;
        \item Avaliação dos esforços induzidos no obstáculo submerso;
        \item Variações de vazão na entrada do domínio e seus efeitos na dinâmica do escoamento;
        % \item Cálculo dos números adimensionais para caracterizar o escoamento ao longo do domínio;
        \item Números de Reynolds e de Froude para caracterizar efeitos viscosos.
    \end{itemize}
\end{frame}