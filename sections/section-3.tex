\section{Reologia}

\subsection{Fluidos não newtonianos}
\begin{frame}{Fluidos não newtonianos}
    \includegraphics[width=\textwidth]{../dissertation-project/fig/diagrams/fluid_classification.pdf}
    \captionof{figure}{Fluidos não newtonianos. (a) Curvas se fluidos
    viscoplásticos. (b) Fluidos Puramente Viscosos. (c) Teste de taxa de deformação
    constante.}
\end{frame}

\subsection{Equacionamento}
\begin{frame}
    \begin{exampleblock}{Conservação da Massa}
        \begin{equation}
            \partialderiv{\rho}{t} + \nabla \cdot \rho \bm{u} = 0
        \end{equation}
    \end{exampleblock}

    \begin{exampleblock}{Conservação de \textit{Momentum}}
        \begin{equation}
            \rho \left( \partialderiv{\bm{u}}{t}  + \bm{u} \cdot \nabla \bm{u} \right) =
            \bm{b} + \nabla \cdot \bm{T}
        \end{equation}
    \end{exampleblock}

    \hspace{0.5cm} Definindo o tensor de tensão como $\bm{T} = -p\bm{I} + \bm{\tau}$ e a 
    aceleração gravitacional como a única força externa aplicada ao fluido:
        \vspace{0.25cm}
        \begin{equation}
            \rho \left( \partialderiv{\bm{u}}{t}  + \bm{u} \cdot \nabla \bm{u} \right) =
            -\nabla p + \nabla \cdot \bm{\tau} + \rho \bm{g}
        \end{equation}
\end{frame}


\subsection{Fluido HB em Plano Inclinado}

\begin{frame}{Fluido HB em Plano Inclinado}
    \begin{minipage}[c]{0.49\textwidth}
        \begin{exampleblock}{Plano inclinado, bidimensional}
            \includegraphics[width=\textwidth]{../dissertation-project/fig/diagrams/free-surface_inclined-plane.pdf}
            \captionof{figure}{Escoamento de superfície livre de altura $h$ em um plano infinitamente largo, com
            velocidade uniforme.}
        \end{exampleblock}
    \end{minipage}
    \hfill
    \begin{minipage}[c]{0.49\textwidth}
        \begin{exampleblock}{Plano inclinado, "tridimensional"}
            \includegraphics[width=\textwidth]{../dissertation-project/fig/diagrams/free-surface-obstacle.pdf}
            \captionof{figure}{Escoamento incidindo sobre um corpo submerso retangular ao fundo do canal, com velocidade $u(y)$.}
        \end{exampleblock}
    \end{minipage}
\end{frame}

\begin{frame}
    \begin{exampleblock}{Modelo de fluido tipo HB}
        \begin{equation}
            \label{eq:HB_shear_stress}
            \begin{split}
                \tau &= \yieldstress + K \shearrate^n
                \text{ ,\hspace{0.5 cm} quando \hspace{0.2cm}} \shearrate = \partialderiv{u}{y} \ne 0
                \\
                |\tau| &\leq \yieldstress
                \text{ ,\hspace{1.65cm} quando \hspace{0.2cm}} \shearrate = 0
            \end{split}
        \end{equation}
    \end{exampleblock}
\end{frame}

\begin{frame}
    \hspace{0.5cm} Em regime permanente, a equação de \textit{momentum} fica:
    \begin{equation}
        \begin{split}
            \partialderiv{\tau}{y} + \rho g \sin{\theta} &= 0 \\
            -\partialderiv{p}{y} - \rho g \cos{\theta} &= 0
        \end{split}
        \label{eq:motion_steady_state_partial}
    \end{equation}

    \hspace{0.5cm} Integrando e usando as condições de contorno da superfície livre e não deslizamento na parede:
    \begin{equation}
        \begin{split}
            u(y) &= \frac{\alpha}{m+1} \left[y_0^{(m+1)} - {(y_0-y)}^{(m+1)}\right]
            \hspace{0.5cm} \text{quando} \hspace{0.5cm} y \leq y_0
            \\
            u(y) &= u(y_0) = \frac{\alpha}{m+1} \left[ y_0^{(m+1)} \right]
            \hspace{1.9cm} \text{quando} \hspace{0.5cm} h \geq y \geq y_0
        \end{split}
        \label{eq:velocity_distribution}
    \end{equation}
    
    com $m=\frac{1}{n}$, $y_0 = h - \frac{\yieldstress}{\rho g \sin{\theta}}$ e $\alpha = \left({\frac{\rho g \sin{\theta}}{K}}\right)^m$.
\end{frame}

\begin{frame}
    \begin{exampleblock}{Perfil de velocidade de um fluido tipo HB}
        \centering
        \includegraphics[width=0.55\textwidth]{../dissertation-project/fig/diagrams/HB_velocity-distribution_inclined-plane.pdf}
        \captionof{figure}{Distribuição de velocidade ao longo da seção transversal, para diferentes valores de $n$. Adaptado de \cite{coussot_1997_mudflow}.}
    \end{exampleblock}
\end{frame}