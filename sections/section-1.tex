\section{Introdução}

\subsection{Contextualização}

\begin{frame}{Contextualização}
    \begin{figure}
        \centering
        \includegraphics[width=0.5\textwidth]{../dissertation-project/fig/disasters_data/world_2000-2023_riskType.pdf}
        \caption{Quatro principais tipos de risco, em número de ocorrências, envolvendo
        desastres naturais, anos de 1900 a 2023. Dados de \cite{emdat_2023}.}
    \end{figure}
\end{frame}


\begin{frame}{Contextualização}
    \begin{figure}
        \centering
        \includegraphics[width=0.5\textwidth]{../dissertation-project/fig/disasters_data/world_accident_temperature_year.pdf}
        \caption{Número de deslizamento de terra e média da temperatura global. Dados de \cite{emdat_2023,NASA/GISS}}
    \end{figure}
\end{frame}

\begin{frame}{Contextualização}
    \begin{figure}
        \centering
        \includegraphics[width=0.75\textwidth]{../dissertation-project/fig/geospatial_maps/world_2000-2023_accidents_country_filtered.pdf}
        \caption{Número de desastres hidrológicos e geofísicos por país, anos 2000 a 2023. Dados de \cite{emdat_2023}}
    \end{figure}
\end{frame}

\begin{frame}{Contextualização}
    \begin{figure}
        \centering
        \includegraphics[width=0.6\textwidth]{../dissertation-project/fig/geospatial_maps/brazil_2000-2023_accidents_subtype_uf.pdf}
        \caption{Movimentos de Massa por estados no Brasil, anos 2000 a 2023. Dados de \cite{emdat_2023}}
    \end{figure}
\end{frame}

\begin{frame}{Contextualização}
    \begin{figure}
        \centering
        \begin{subfigure}{0.45\textwidth}
            \centering
            \includegraphics[width=\textwidth]{../dissertation-project/fig/disasters_data/rj_month_rain_massMovement.pdf}
            \caption{Rio de Janeiro.}
            \label{fig:rj}
        \end{subfigure}
        \hfill
        \begin{subfigure}{0.45\textwidth}
            \centering
            \includegraphics[width=\textwidth]{../dissertation-project/fig/disasters_data/sp_month_rain_massMovement.pdf}
            \caption{São Paulo.}
            \label{fig:sp}
        \end{subfigure}
    
        \caption{Movimentos de massa e índice pluviométrico. Dados dos acidentes de \cite{atlas_brazil_2023}, anos de 1931 a 2023. Dados climatológicos de \cite{ANA}, anos de 1931 a 2020.}
        \label{fig:test}
    \end{figure}
\end{frame}




% ---------------------------------------------------------------------------------------------------- %
\subsection{Objetivos}

\begin{frame}{Objetivos}
asdf
\end{frame}
% ---------------------------------------------------------------------------------------------------- %

\subsection{Organização do Trabalho}

\begin{frame}{Frame Title}
    \begin{itemize}
        \item This template is modified from Tsinghua University's Beamer template: \url{https://www.overleaf.com/latex/templates/thu-beamer-theme/vwnqmzndvwyb} \pause
        \item The original template is modified from \newline \url{https://www.latexstudio.net/archives/4051.html}
        \item The real original template is not found \cite{origin}.
    \end{itemize}
\end{frame}