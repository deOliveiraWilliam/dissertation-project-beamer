\section{Introdução}

\subsection{Contextualização}

\begin{frame}{Contextualização}
    \begin{figure}
        \centering
        \includegraphics[width=0.5\textwidth]{../dissertation-project/fig/disasters_data/world_2000-2023_riskType.pdf}
        \caption{Quatro principais tipos de risco, em número de ocorrências, envolvendo
        desastres naturais, anos de 1900 a 2023. Dados de \cite{emdat_2023}.}
    \end{figure}
\end{frame}


\begin{frame}{Contextualização}
    \begin{figure}
        \centering
        \includegraphics[width=0.5\textwidth]{../dissertation-project/fig/disasters_data/world_accident_temperature_year.pdf}
        \caption{Número de deslizamento de terra e média da temperatura global. Dados de \cite{emdat_2023,NASA/GISS}}
    \end{figure}
\end{frame}

\begin{frame}{Contextualização}
    \begin{figure}
        \centering
        \includegraphics[width=0.75\textwidth]{../dissertation-project/fig/geospatial_maps/world_2000-2023_accidents_country_filtered.pdf}
        \caption{Número de desastres hidrológicos e geofísicos por país, anos 2000 a 2023. Dados de \cite{emdat_2023}}
    \end{figure}
\end{frame}

\begin{frame}{Contextualização}
    \begin{figure}
        \centering
        \includegraphics[width=0.6\textwidth]{../dissertation-project/fig/geospatial_maps/brazil_2000-2023_accidents_subtype_uf.pdf}
        \caption{Movimentos de Massa por estados no Brasil, anos 2000 a 2023. Dados de \cite{emdat_2023}}
    \end{figure}
\end{frame}

\begin{frame}{Contextualização}
    \begin{figure}
        \centering
        \begin{subfigure}{0.45\textwidth}
            \centering
            \includegraphics[width=\textwidth]{../dissertation-project/fig/disasters_data/rj_month_rain_massMovement.pdf}
            \caption{Rio de Janeiro.}
            \label{fig:rj}
        \end{subfigure}
        \hfill
        \begin{subfigure}{0.45\textwidth}
            \centering
            \includegraphics[width=\textwidth]{../dissertation-project/fig/disasters_data/sp_month_rain_massMovement.pdf}
            \caption{São Paulo.}
            \label{fig:sp}
        \end{subfigure}
        \caption{Movimentos de massa e índice pluviométrico. Dados dos acidentes de \cite{atlas_brazil_2023}, anos de 1931 a 2023. Dados climatológicos de \cite{ANA}, anos de 1931 a 2020.}
        \label{fig:test}
    \end{figure}
\end{frame}


\begin{frame}
    \begin{minipage}[c]{0.6\textwidth}
        \centering
        \includegraphics[width=\textwidth]{../dissertation-project/fig/svgs/brazil_disasters_pictures.pdf}
    \end{minipage}
    \hfill
    \begin{minipage}[c]{0.38\textwidth}
        \captionof{figure}{
    (a) Inundação em Itapetinga, Bahia \cite{CNN_bahia}.
    (b) Deslizamento de terra em Petrópolis, RJ \cite{CNN_landslides}.
    (c) Deslizamento de terra em Recife, PE \cite{G1_recife}. 
    (d) Inundações após chuvas intensas \cite{CNN_floods}. 
    (e) Exposição de encosta após deslizamento \cite{AlJazeera}. 
    (f) Deslizamento de terra, \cite{Ag_Brasil}. 
    (g) Deslizamento de terra próximo à divisa entre Paraná e Santa Catarina \cite{national_news}. 
    (h) Movimento de massa. Chuvas intensas e declividade elevada. \cite{G1_gramado1}.}
    \end{minipage}
\end{frame}

% ---------------------------------------------------------------------------------------------------- %
\subsection{Objetivos}

\begin{frame}
    \begin{exampleblock}{Objetivos}
        \begin{itemize}
            \item Simulação numérica;\pause
            \item Resultado experimental (?);\pause
            \item Escoamento de superfície livre;\pause
            \item Fluido viscoplástico;\pause
            \item Comportamento do escoamento sobre obstáculos;\pause
            \item Regime de escoamento e regime hidráulico; \pause
            \item Diferentes configurações de obstruções;\pause
            \item Observação de diferentes modelos reológicos.
        \end{itemize}
    \end{exampleblock}
\end{frame}
% ---------------------------------------------------------------------------------------------------- %

\subsection{Organização do Trabalho}

\begin{frame}
    \begin{minipage}[c]{0.49\textwidth}
        \begin{exampleblock}{Definição do Problema}
            \begin{itemize}
                \item Movimentos geofísicos de massa;
                \item Escoamento em canais;
                \item Escoamento de superfície livre.
            \end{itemize}            
        \end{exampleblock}
        \hfill
        \pause
        \begin{exampleblock}{Reologia}
            \begin{itemize}
                \item Fluidos não newtonianos e suas propriedades; 
                \item Modelo de fluido do tipo Herschel-Bulkley;
                \item Equacionamento.
            \end{itemize}
        \end{exampleblock}
    \end{minipage}    
    \hfill
    \begin{minipage}[c]{0.49\textwidth}
        \pause
        \begin{exampleblock}{Escoamento Sobre Estruturas}
            \begin{itemize}
                \item Interação entre o fluido e as estruturas submersas.
            \end{itemize}
        \end{exampleblock}
        \hfill

        \pause
        \begin{exampleblock}{Metodologia Numérica}
            \begin{itemize}
                \item Solução de superfície livre;
                \item Modelo bi-viscoso;
                \item Geração das geometrias e malhas;
                \item Condições de contorno.
            \end{itemize}    
        \end{exampleblock}
    \end{minipage}
\end{frame}